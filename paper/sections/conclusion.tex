\section{Conclusion}

Schr\"odinger cat states, i.e. quantum superpositions of coherent states, present a compelling opportunity for the implementation of error-protected qubits.
In particular, these cat states can be created in a superconducting microwave resonator with a Kerr nonlinearity and subsequently used as the logical qubit basis states in a quantum error correction scheme.
Individual cat-qubits can be adiabatically initialized and subsequently stabilized by applying a two-photon squeezing drive, as demonstrated by Grimm et al. \cite{grimm_2020}
Furthermore, universal control of these qubits is straightforward, requiring only the application of a coherent microwave drive or the turning off of the squeezing drive.

An obvious extension to the operation of this Kerr-cat qubit is to capacitively couple two of them together, as proposed by e.g. Mirrahimi et al. \cite{mirrahimi_2014}.
Under this capacitive coupling the two-qubit system becomes entangled, allowing for the initialization and control of arbitrary two-qubit states.

In this paper we have numerically simulated the initialization and operation of single- and two-qubit systems composed of Kerr-cat qubits, using the experimentally measured system parameters determined by Grimm et al. \cite{grimm_2020} where possible.
Using these simulations we have replicated some of the experimental results, including the operation of single-qubit gates.
We have also simulated the evolution of the system under the capacitive coupling Hamiltonian and demonstrated that it can entangle the qubits in a time far shorter than these qubits' decoherence times.
Finally, we have demonstrated that measurements of the two-qubit $Q$ function could in principle be used to distinguish between bipartite entanglement and statistical mixtures, though more work is needed to confirm that this is experimentally feasible for the particular system at hand.

