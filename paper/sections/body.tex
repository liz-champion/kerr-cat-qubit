\section{\label{sec:init} Initialization and stabilization of a Kerr-cat qubit}

\begin{figure}[t]
    \centering
    \includegraphics[width=\columnwidth]{figures/initialization.pdf}
    \caption{Initialization of the $\ket{\mathcal{C}_\alpha^+}$ cat state. The projection of the resonator state onto the target state is shown in black, while the squeezing drive envelope is shown in red.}
    \label{fig:initialization}
\end{figure}

\begin{figure}[t]
    \centering
    \includegraphics[width=\columnwidth]{figures/decay.pdf}
    \caption{Decay of the logical qubit due to decoherence. The red dashed line shows an exponential fit to the data after \SI{0.4}{\micro\s}, yielding a decay time of $\tau_Z =$ \SI{2.526}{\micro\s} $\pm$ \SI{0.002}{\micro\s}.}
    \label{fig:decay}
\end{figure}

\begin{figure*}[t]
     \centering
     \begin{subfigure}[b]{0.32\textwidth}
         \centering
         \includegraphics[width=\textwidth]{figures/ground.pdf}
         \caption{}
         \label{fig:ground}
     \end{subfigure}
     \hfill
     \begin{subfigure}[b]{0.32\textwidth}
         \centering
         \includegraphics[width=\textwidth]{figures/initialized_cat.pdf}
         \caption{}
         \label{fig:initialized_cat}
     \end{subfigure}
     \hfill
     \begin{subfigure}[b]{0.32\textwidth}
         \centering
         \includegraphics[width=\textwidth]{figures/ideal_cat.pdf}
         \caption{}
         \label{fig:ideal_cat}
     \end{subfigure}
        \caption{Wigner functions for (a) the initial thermal state, (b) the adiabatically initialized $\ket{0_L} = \ket{\mathcal{C}_\alpha^+}$ cat state, and (c) the ideal cat state.}
        \label{fig:initialization_wigner}
\end{figure*}

Cat states can be initialized and stabilized in a superconducting microwave resonator by introducing a Kerr nonlinearity and a two-photon squeezing drive, as proposed by Puri et al. and later realized experimentally by Grimm et al. \cite{puri_2017,grimm_2020}.
The experimental platform used by Grimm et al. consists of a nonlinear resonator place within a 3D microwave cavity; the resulting nonlinearity facilitates the generation of a two-photon drive through three-wave mixing and is also responsible for the Kerr nonlinearity.
See \cite{grimm_2020}, particularly the supplementary information, for details.

The Hamiltonian describing this system in the absence of any other microwave drives (after a transformation to the frame rotating at the resonator frequency) is
\begin{equation}
    \hat{H}_{\rm sys} / \hbar = -K \hat{a}^{\dag 2} \hat{a}^2 + \epsilon_2 \hat{a}^{\dag 2} + \epsilon_2^* \hat{a}^2,
\end{equation}
where $K$ is the strength of the Kerr nonlinearity and $\epsilon_2$ is the two-photon drive strength.

This Hamiltonian has two degenerate lowest-energy eigenstates, specifically the coherent states $\ket{\pm \alpha}$ for $\alpha = \sqrt{\epsilon_2 / K}$.
Note that this Hamiltonian has the property of preserving photon number parity: photons are only created or destroyed two at a time.
In the absence of the two-photon drive, the lowest-energy eigenstate is simply the $\ket{0}$ Fock state, which has even photon number parity.
By ramping the squeezing drive on and off slowly with respect to $1/2K$, the $\ket{0}$ Fock state is adiabatically mapped onto the logical basis state $\ket{0_L} = \ket{\mathcal{C}_\alpha^+}$, due to the parity-preserving nature of the Hamiltonian.
This is the method demonstrated experimentally by Grimm et al. \cite{grimm_2020}.

Figure \ref{fig:initialization} shows a numerical simulation of this process.
The resonator is assumed to be initially in a thermal state with a population of $n_{\rm th} = 0.04$ in the $\ket{1}$ Fock state.
The strength of the Kerr nonlinearity is $K / 2\pi =$ \SI{6.7}{\MHz} and the two-photon drive is ramped up with a $\tanh$ pulse shape over a period of \SI{320}{\ns} to a final value of $\epsilon_2 / 2 \pi =$ \SI{15.5}{\MHz} (all system parameters are taken from Grimm et al. \cite{grimm_2020}).
The time evolution of the system is determined using the Lindblad master equation \cite{manzano_2020}:
\[
    \dot{\hat{\rho}} = - \frac {i} {\hbar} [\hat{H}_{\rm sys}, \hat{\rho}] + \kappa_a (1 + n_{\rm th}) \mathcal{D}[\hat{a}] \hat{\rho} + \kappa_a n_{\rm th} \mathcal{D}[\hat{a}^\dag] \hat{\rho}
\]
where $\kappa_a = 1/T_1$, $T_1 =$ \SI{15.5}{\micro\s} is the single-photon loss rate of the resonator and
\[
    \mathcal{D}[\hat{\mathcal{O}}] \hat{\rho} = \hat{\mathcal{O}} \hat{\rho} \hat{\mathcal{O}}^\dag - \frac {1} {2} \hat{\mathcal{O}}^\dag \hat{\mathcal{O}} \hat{\rho} - \frac {1} {2} \hat{\rho} \hat{\mathcal{O}}^\dag \hat{\mathcal{O}}.
\]

Figure \ref{fig:initialization_wigner} shows the Wigner functions for the initial thermal state, adiabatically initialized cat state, and the ideal cat state, demonstrating the effectiveness of the adiabatic mapping in initializing the cat qubit.
The imperfect initialization (visible in Figure \ref{fig:initialization}) is primarily due to the nonzero population of the $\ket{1}$ Fock state in the resonator's initial thermal state and photon loss in the resonator.

Figure \ref{fig:decay} shows the decay of the qubit's $\langle Z \rangle$-component when initialized to the $\ket{\mathcal{C}_\alpha^+}$ state.
An exponential fit to this curve yields a decay time of $\tau_Z =$ \SI{2.526}{\micro\s} $\pm$ \SI{0.002}{\micro\s}; this is close to, albeit just outside the uncertainty of, the experimentally measured value of $\tau_Z =$ \SI{2.60}{\micro\s} $\pm$ \SI{0.07}{\micro\s} reported by Grimm et al. \cite{grimm_2020}.

\section{\label{sec:gates} Single-qubit gate operations}

Having characterized the initialization and subsequent decoherence of a Kerr-cat qubit, we now turn our attention to a universal single-qubit control, realized in an $X(\theta)$ gate for arbitrary $\theta$ and a $Z(\pi/2)$ gate.
In particular, we wish to show that these gates can be performed in a time much shorter than the qubit's decoherence time.

\subsection{Arbitrary $X$ rotations}

One can show that the application of an additional coherent drive induces Rabi oscillations in the encoded qubit: in particular, the Hamiltonian
\begin{equation} \label{eq:x}
    \hat{H}_x / \hbar = \epsilon_x \hat{a}^\dag + \epsilon_x^* \hat{a},
\end{equation}
when projected onto the subspace spanned by $\{ \ket{\mathcal{C}_\alpha^+}, \ket{\mathcal{C}_\alpha^-}\}$, has the form \cite{grimm_2020,mirrahimi_2014}
\[
    \Omega_x \sigma_x^L
\]
where the superscript $L$ indicates that we are working in the logical qubit basis and
\[
    \Omega_x = {\rm Re}(2 \epsilon_x \sqrt{\bar{n}}).
\]
Evolution of the cat qubit under this Hamiltonian has the effect of inducing Rabi oscillations, much like a two-level system under the same coherent drive.
One can gain some intuition for this interaction by considering the effect of Eq. (\ref{eq:x}) on the photon number parity: the exchange of single photons caused by this drive has the effect of swapping photon number parity, corresponding to transitions between $\ket{\mathcal{C}_\alpha^+}$ and $\ket{\mathcal{C}_\alpha^-}$.

In Figure \ref{fig:rabi} we show the time evolution of the cat qubit using a relatively low drive strength of $\epsilon_x / 2 \pi =$ \SI{170}{\kHz}.
The qubit is clearly undergoing Rabi oscillations.
Note, however, that these Rabi oscillations differ from those of a two-level system in an important way: the oscillation frequency varies with the amplitude and phase of the drive.
When using this effect to implement the $X(\theta)$ gate, we can therefore choose a large drive amplitude to make the gate time very short.
Evolution of the cat qubit under the coherent drive for a time $t_x = \theta / 2 \Omega_x$ effects an $X(\theta)$ gate on the qubit; this gate is demonstrated in Figure \ref{fig:x_gate} for a rotation by $\theta = \pi$ using a drive strength of $\epsilon_x / 2 \pi =$ \SI{6}{\MHz}.

\begin{figure}[t]
    \centering
    \includegraphics[width=\columnwidth]{figures/rabi.pdf}
    \caption{Rabi oscillations driven by a coherent resonant drive.}
    \label{fig:rabi}
\end{figure}

\subsection{The $Z(\pi/2)$ gate}

Having implemented rotations by an arbitrary angle about the $X$-axis of the cat-qubit's Bloch sphere, all that is needed to reach any point on the sphere is a $\pi/2$ rotation about the $Z$-axis.
This gate is realized as the evolution of the cat-qubit under the Kerr Hamiltonian in the absence of any drives.
In particular, this evolution for a time $t_z = \pi/2K$ maps the coherent states $\ket{\pm \alpha}$ onto $(\ket{\pm \alpha} - i \ket{\mp \alpha}) / \sqrt{2}$, thus inducing a rotation by $\pi/2$ about the $Z$-axis of the Bloch sphere \cite{grimm_2020,mirrahimi_2014}.

Figure \ref{fig:bloch} shows the projection of the resonator state onto the Bloch states $\ket{\pm X}$, $\ket{\pm Y}$, and $\ket{\pm Z}$ under the application of an $X(\pi/2)$ gate followed by a $Z(\pi/2)$ gate.
We see that the qubit is first initialized in the $\ket{+Z} = \ket{\mathcal{C}_\alpha^+}$ state, then rotated about the $X$-axis onto the $\ket{-Y} = (\ket{\mathcal{C}_\alpha^+} - i \ket{\mathcal{C}_\alpha^-}) / \sqrt{2}$ state.
Following this it is rotated by $\pi/2$ about the $Z$-axis onto the $\ket{+X} = (\ket{\mathcal{C}_\alpha^+} + \ket{\mathcal{C}_\alpha^-}) / \sqrt{2}$ state, as we would expect.
Through sequences of $X(\theta)$ and $Z(\pi/2)$ gates we could in principle reach any point on the Bloch sphere.

\begin{figure}[t]
    \centering
    \includegraphics[width=\columnwidth]{figures/x_gate.pdf}
    \caption{Bit flip due to the application of an $X(\pi)$ gate.}
    \label{fig:x_gate}
\end{figure}


\begin{figure}[t]
    \centering
    \includegraphics[width=\columnwidth]{figures/bloch.pdf}
    \caption{Projection of the cat-qubit state onto the states corresponding to the axes of the Bloch sphere during state initialization and the application of an $X(\pi/2)$ gate followed by a $Z(\pi/2)$ gate. Solid lines correspond to the positive direction on the Bloch sphere axis axis while dashed lines correspond to the negative direction.}
    \label{fig:bloch}
\end{figure}

\section{\label{sec:two_qubit} The two-qubit entangling gate}

While Grimm et al. \cite{grimm_2020} is concerned with the stabilization and operation of a single qubit, one can imagine capacitively coupling two such resonators, resulting in the coupling Hamiltonian
\begin{equation}
    \hat{H}_c / \hbar = \chi(t) (\hat{a}_1^\dag \otimes \hat{a}_2 + \hat{a}_1 \otimes \hat{a}_2^\dag)
\end{equation}
where $\chi(t)$ is the coupling strength and the operator subscripts denote the subsystem on which the operator acts.
For simplicity we assume the coupling can be turned on and off, which might be realized experimentally by bringing the qubits in and out of resonance with one another or through some more complex coupling scheme \cite{wendin_2017}.
We further assume that both resonators have identical physical parameters.
We now have a four-dimensional logical qubit basis spanned by the states $\ket{\mathcal{C}_\alpha^\pm} \otimes \ket{\mathcal{C}_\alpha^\pm}$, which for notational simplicity will henceforth be denoted $\{\ket{00}, \ket{01}, \ket{10}, \ket{11}  \}$ (not to be confused with Fock states).

\begin{figure}[t]
    \centering
    \includegraphics[width=\columnwidth]{figures/bell.pdf}
    \caption{Bell state generation gate sequence. We plot both the projection of the two-qubit state onto each of the logical basis states (upper) and the projection onto the Bell states (lower).}
    \label{fig:bell}
\end{figure}


\begin{figure*}[t]
     \centering
     \begin{subfigure}[b]{0.32\textwidth}
         \centering
         \includegraphics[width=\textwidth]{figures/q_mixed.pdf}
         \caption{}
         \label{fig:ground}
     \end{subfigure}
     \hfill
     \begin{subfigure}[b]{0.32\textwidth}
         \centering
         \includegraphics[width=\textwidth]{figures/q_entangled.pdf}
         \caption{}
         \label{fig:initialized_cat}
     \end{subfigure}
     \hfill
     \begin{subfigure}[b]{0.32\textwidth}
         \centering
         \includegraphics[width=\textwidth]{figures/q_actual.pdf}
         \caption{}
         \label{fig:ideal_cat}
     \end{subfigure}
        \caption{Husimi $Q$ functions for (a) a mixed state, (b) an ideal entangled state, and (c) the result of applying the entangling gate followed by a $Z(\pi/2)$ gate. The Husimi $Q$ function is in reality a function of two complex numbers, but here we show a two-dimensional representation restricted to real values of $\alpha_1$ and $\alpha_2$. The secondary peaks visible on the result of the simulation (c) are an artifact of the finite-dimensional Hilbert space ($n=0,\ldots,10$ for each resonator). Running the simulation in a higher-dimensional Hilbert space would remove these, but the increased computational cost makes this infeasible.}
        \label{fig:Q}
\end{figure*}

When projected onto the subspace spanned by this basis, the coupling Hamiltonian takes the form \cite{mirrahimi_2014}
\[
    \Omega_c \sigma_{x,1}^L \otimes \sigma_{x,2}^L
\]
where
\[
    \Omega_c = 2 \bar{n} \chi(t).
\]
Evolution of the two-qubit system under this Hamiltonian for a time $t_c = \pi/4\Omega_c$ has the effect of entangling the cat-qubits: this entangling gate can be expressed as the unitary operation
\[
    E^L = \frac {1} {\sqrt{2}} \begin{bmatrix}
        1 & 0 & 0 & -i \\
        0 & 1 & -i & 0 \\
        0 & -i & 1 & 0 \\
        -i & 0 & 0 & 1
    \end{bmatrix},
\]
which acts on the logical (encoded) qubit rather than the resonator Fock states.

By composing this two-qubit entangling gate with $X$ and $Z$ gates applied to individual qubits we can create Bell states in the logical qubit space.
For example, Figure \ref{fig:bell} demonstrates the creation of a $\ket{\Phi^+}$ Bell state by applying the entangling gate to the two-qubit system followed by applying a $Z(\pi/2)$ gate to the first qubit.
This is represented in both the logical qubit basis and the Bell state basis.

In addition to plotting the projection of the two-qubit state onto the target Bell state, as in Figure \ref{fig:bell}, we can also attempt to confirm the presence of entanglement by visualizing the phase space of the two-qubit system.
Because the Wigner function for a bipartite system is difficult to compute, we instead compute the Husimi $Q$ function, defined in the case of a bipartite system as
\begin{equation} \label{eq:Q}
    Q(\alpha_1, \alpha_2) = \bra{\alpha_1} \otimes \bra{\alpha_2} \hat{\rho} \ket{\alpha_1} \otimes \ket{\alpha_2}.
\end{equation}
In principle this is a quasiprobability distribution over four real numbers, since $\alpha_1$ and $\alpha_2$ are complex.
For simplicity and ease of plotting we take $\alpha_1, \alpha_2 \in \mathbb{R}$.

For a separable state, i.e. $\hat{\rho} = \hat{\rho}_1 \otimes \hat{\rho}_2$, the $Q$ function factorizes \cite{floerchinger_2022}:
\[
    Q(\alpha_1, \alpha_2) = Q_1(\alpha_1) Q_2(\alpha_2).
\]
Suppose the $Q$ function for such a separable state in our two-dimensional representation has peaks at $(\alpha_1, \alpha_2)$ and $(-\alpha_1, -\alpha_2)$.
The above factorization implies that the $Q$ function must have peaks at $(-\alpha_1, \alpha_2)$ and $(\alpha_1, -\alpha_2)$ as well, while this need not be the case for a non-separable (i.e. entangled) state.
Finally, note that by Eq. (\ref{eq:Q}) the $Q$ function of a statistical mixture is the sum of the $Q$ functions of its constituent parts.

Figure \ref{fig:Q} shows the $Q$ functions for the statistical mixture $\hat{\rho} = (\ket{00}\bra{00} + \ket{11}\bra{11}) / 2$, the entangled state $\ket{\Phi^+} = (\ket{00} + \ket{11}) / \sqrt{2}$ in the ideal case, and the same entangled state as initialized in the numerical simulation.
Note that the statistical mixture and the entangled state both have the same measurement probabilities in the logical qubit basis, and are therefore indistinguishable when only considering such measurements.
The $Q$ functions, however, demonstrate the presence of bipartite entanglement: the statistical mixture obeys the above requirement regarding the locations of its peaks, while the entangled state does not.
Because $Q$ functions are experimentally accessible \cite{kirchmair_2013,floerchinger_2022}, measurements of it could possibly be used to confirm the presence of entanglement in an experimental implementation of the coupling described here.
